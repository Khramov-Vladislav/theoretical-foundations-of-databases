\documentclass[a4paper, 14pt]{extarticle} % 14pt вместо 12pt
\usepackage[T2A]{fontenc}
\usepackage[utf8]{inputenc}
\usepackage[russian]{babel}
\usepackage{newtxtext,newtxmath} % Times New Roman шрифт
\usepackage{geometry}
\usepackage{tabularx}
\usepackage{multirow}
\usepackage{booktabs}
\usepackage{amsmath}
\usepackage{graphicx}
\usepackage{float}
\usepackage{listings}
\usepackage{xcolor}
\usepackage{setspace} % Для межстрочного интервала
\usepackage{indentfirst} % Отступ первой строки абзаца

% Настройка страницы по ГОСТ
\geometry{
    left=30mm,    % 3 см
    right=15mm,   % 1,5 см
    top=20mm,     % 2 см
    bottom=20mm,  % 2 см
}

% Межстрочный интервал 1,5
\onehalfspacing

% Отступ первой строки абзаца 1,25 см
\setlength{\parindent}{1.25cm}

% Настройка стиля листингов
\lstset{
    language=C++,
    basicstyle=\ttfamily\small, % немного меньше основного текста
    keywordstyle=\color{blue}\bfseries,
    commentstyle=\color{green!60!black},
    stringstyle=\color{red},
    numbers=left,
    numberstyle=\tiny\color{gray},
    stepnumber=1,
    numbersep=5pt,
    tabsize=4,
    showstringspaces=false,
    breaklines=true,
    frame=single,
    backgroundcolor=\color{gray!5},
    rulecolor=\color{gray!30},
    captionpos=b,
    aboveskip=10pt,
    belowskip=10pt
}

% Настройка заголовков
\usepackage{titlesec}

% Формат заголовков разделов (section)
\titleformat{\section}
    {\normalfont\Large\bfseries}  % размер: Large, шрифт: жирный
    {\thesection}                  % номер раздела
    {1em}                          % расстояние между номером и названием
    {}                             % код после заголовка

% Формат подразделов (subsection)
\titleformat{\subsection}
    {\normalfont\large\bfseries}   % размер: large, шрифт: жирный
    {\thesubsection}
    {1em}
    {}

% Формат под-подразделов (subsubsection)
\titleformat{\subsubsection}
    {\normalfont\normalsize\bfseries} % размер: normal, шрифт: жирный
    {\thesubsubsection}
    {1em}
    {}

% Установка отступов до и после заголовков
\titlespacing*{\section}{0pt}{12pt}{6pt}      % {влево}{до}{после}
\titlespacing*{\subsection}{0pt}{10pt}{4pt}
\titlespacing*{\subsubsection}{0pt}{8pt}{2pt}

% Убираем отступы в списках
\usepackage{enumitem}
\setlist{nosep, leftmargin=1.25cm}

% Для не нумерованных заголовков тоже
\titleformat{name=\section,numberless}
    {\normalfont\Large\bfseries}
    {}
    {0pt}
    {}

\titleformat{name=\subsection,numberless}
    {\normalfont\large\bfseries}
    {}
    {0pt}
    {}

% Настройка гиперссылок (рекомендуется подключать последним)
\usepackage{hyperref}
\hypersetup{
    colorlinks=true,
    linkcolor=blue,
    filecolor=magenta,      
    urlcolor=cyan,
    pdftitle={Отчет по лабораторной работе №2},
    pdfauthor={Храмов В.Д.}
}

%%%%%%%%%%%%%%%%%%%%%%%%%%%%%%%%%%%%%%%%%%%%%%%%%%%%%%%%%%%%%%%

\begin{document}

    \begin{titlepage}
        \thispagestyle{empty}
        
        \begin{center}
        
            { \large МИНИСТЕРСТВО НАУКИ И ВЫСШЕГО ОБРАЗОВАНИЯ \\ 
                РОССИЙСКОЙ ФЕДЕРАЦИИ }
             
            { «Санкт-Петербургский политехнический университет Петра Великого» }
            
            { Институт компьютерных наук и кибербезопасности \\ 
                Высшая школа технологий искусственного интеллекта \\
                Направление: 02.03.01. «Математика и компьютерные науки» \\ }        
              
              \vspace{3cm}
            { \large Отчет о выполнении курсовой работы \\ 
              По дисциплине: «Теоретические основы баз данных» \\ 
              На тему: «Учет проживающих в общежитии» }
    
        \end{center}
    
        \vspace{3cm}

        \begin{center}
            % Студент
            \noindent
            \begin{minipage}[t]{\textwidth}
                \raggedright Студент  \\ группы 5130201/40002 \hfill 
                \underline{\hspace{3cm}} Храмов В.Д.
            \end{minipage}
            
            \vspace{1cm}
            
            % Преподаватель
            \noindent
            \begin{minipage}[t]{\textwidth}
                \raggedright Преподаватель \hfill 
                \underline{\hspace{3cm}}  Попов С.Г.
            \end{minipage}
        \end{center}
        
        \vspace{1cm}
        
        % Дата справа
        \begin{flushright}
            \large
            «\underline{\hspace{1cm}}» \underline{\hspace{3cm}} 2026 г.
        \end{flushright}
        
        \vspace{0.5cm}
        
        \vfill
        
        \begin{center}
            Санкт-Петербург \\
            2026
        \end{center}   
            
    \end{titlepage}
    
    \pagenumbering{arabic} 
    \setcounter{page}{2}  
    
%%%%%%%%%%%%%%%%%%%%%%%%%%%%%%%%%%%%%%%%%%%%%%%%%%%%%%%%%%%%%%%
    \newpage
    
    \section{Реферат}
%%%%%%%%%%%%%%%%%%%%%%%%%%%%%%%%%%%%%%%%%%%%%%%%%%%%%%%%%%%%%%%
    \newpage
    
    \tableofcontents
%%%%%%%%%%%%%%%%%%%%%%%%%%%%%%%%%%%%%%%%%%%%%%%%%%%%%%%%%%%%%%%
    \newpage
    
    \section*{Введение}
    \addcontentsline{toc}{section}{Введение}
%%%%%%%%%%%%%%%%%%%%%%%%%%%%%%%%%%%%%%%%%%%%%%%%%%%%%%%%%%%%%%%
    \newpage
    
    \section{Постановка задачи}
%%%%%%%%%%%%%%%%%%%%%%%%%%%%%%%%%%%%%%%%%%%%%%%%%%%%%%%%%%%%%%%
    \newpage
    
    \section{Раздел «Аналитика»}
        \subsection{Описание предметной области}
            Общежитие -- это жилое помещение, предназначенное для временного проживания людей, которые учатся или работают вдали от постоянного места жительства. Они могут создаваться при университетах или заводах, объединяя людей общим статусом: студенты, аспиранты или рабочие.

            История развития общежитий начинается в средневековье. Первые прообразы общежитий возникли при первых европейских университетах (Болонья, Париж, Оксфорд) в XII–XIII веках. Это были «коллегии» -- пансионы, где студенты не только жили, но и слушали лекции. Проживание было коллективным, подчинялось строгим правилам и дисциплине, что позволяло контролировать поведение студентов и предоставлять им жилье независимо от их финансового положения.

            % Если проводить сравнение с более поздними эпохами, то в индустриальную революцию XVIII–XIX веков появились рабочие казармы и бараки. Они являлись прямыми прародителями современных ведомственных общежитий, поскольку главной целью такого жилья являлось обеспечение заводов рабочей силой, сконцентрированной в одном месте. Такие помещения часто были перенаселены и имели минимальный уровень комфорта, но выполняли ключевую функцию — давали крышу над головой приезжим крестьянам, идущим работать на мануфактуры.

            В России и СССР общежития получили колоссальное развитие после революции 1917 года и в эпоху индустриализации. Они стали символом эпохи: «дом-коммуна» с общими столовыми, прачечными и комнатами для отдыха. Идеология того времени диктовала отказ от индивидуализма в пользу коллективного быта. В советский период сформировался привычный облик студенческого общежития: коридорная или блочная система, комнаты на несколько человек, этажные кухни и душевые, а также строгий контроль со стороны коменданта и воспитателей.
            
            Позже, во второй половине XX века, в мире начали появляться кампусы — университетские городки, где общежития стали частью огромной инфраструктуры: с библиотеками, спортзалами и кафе. Форматы проживания стали разнообразнее: от традиционных блоков до квартирных типов  для семейных студентов или аспирантов. Менялись требования к комфорту, правила проживания становились более демократичными, но основа — предоставление доступного жилья для временного пребывания — оставалась неизменной.

            В наше время общежитие — это не просто место проживания, а целая система, включающая жилые комнаты, обслуживающий персонал и администрацию. Управлением занимается администрация общежития, которая распределяет места, следит за порядком и решает бытовые вопросы жильцов. 
            
            Жизнь в общежитии начинается с заселения. На этом этапе будущий житель подает заявление и предоставляет необходимые документы. Сотрудник отдела кадров или деканата проверяет основание для проживания (приказ о зачислении, направление на работу). Менеджер по заселению (комендант или работник паспортного стола) изучает потребности клиента: статус (студент/аспирант/семейный), пожелания и предлагает варианты расселения. Менеджер также должен учитывать правила внутреннего распорядка и санитарные нормы, чтобы распределение мест соответствовало этим критериям. На этом этапе обсуждается стоимость и условия оплаты. Этот аспект зачастую становится определяющим для проживающих. Житель знакомится с прейскурантом и правилами, а администрация, в свою очередь, разъясняет права и обязанности, которые будут максимально соответствовать данным условиям.
            
            % Далее наступает этап текущего функционирования общежития. В начале заселения делается упор на подписание договора найма и акта приема-передачи помещения, в котором прописывается состояние комнаты и имущества. Менеджер (комендант или завхоз) ведет учет всех возможных расходов, включая оплату коммунальных услуг, текущий ремонт и закупку инвентаря. Помещение (комната) распределяется в зависимости от целей проживания (краткосрочное/долгосрочное), площади, этажа и пожеланий (например, «в тихий блок»). У администрации может иметься множество вариантов расселения (разные блоки, этажи для девушек/юношей). После анализа заявок менеджер выбирает наилучший вариант из списка доступных мест. Работа с подрядчиками (сантехники, электрики, клининговые компании) тоже требует детального подхода и контроля со стороны управляющего. По потребностям жильцов организуется работа столовой или буфета, если они предусмотрены. Компания-поставщик еды должна предоставлять необходимое меню и удовлетворять ограничениям по времени работы и санитарным нормам. Техническое обеспечение (стиральные машины, бойлеры, плиты) выбирается уже в зависимости от нагрузки и потребностей жильцов. Менеджер (завхоз) организует ремонт техники у той компании, которая имеет хорошие отзывы и выгодные цены. Также администрация может договариваться с провайдерами об обеспечении интернетом, что тоже непросто, ведь должны быть учтены потребности сотен жильцов и бюджетные ограничения.
            
            Наличие четкого графика работы персонала (уборщиц, дежурных, сантехников) обеспечивает порядок и структуру работы общежития. Координация процесса эксплуатации здания — это один из самых важных этапов, в котором коммуникация между администрацией и жильцами встает на передний план. Менеджер (комендант) проводит регулярные обходы, встречи с активом или старшими по этажу. Они нужны для обсуждения проблем, обновлений по ремонту и решения возможных конфликтов. А в случае внештатной ситуации (пожар, потоп, отключение электричества) связь становится критически важной, ведь оперативное решение проблем — приоритет дежурного персонала.
            
            % Поддержание порядка и реализация правил проживания — это кульминация ежедневных усилий, приложенных на этапах планирования и организации быта. Ежедневно необходимо проверять чистоту в коридорах, душевых и кухнях, исправность оборудования, наличие воды и отопления. Одной из ключевых задач в процессе контроля является соблюдение тишины и правил пожарной безопасности. Каждое общежитие имеет свой режим (тайминг), и важно строго его придерживаться (время тишины, время закрытия входа). Гибкость в урегулировании конфликтов помогает сохранить комфортную среду и удовлетворить ожидания большинства жильцов.
            
            Периодически (при выселении или ежегодно) администрация оценивает состояние комнат и работу подрядчиков, чтобы, в случае если уборка или ремонт были выполнены плохо, больше не обращаться к ним за услугами или взыскать ущерб с жильца. Также комендант принимает обратную связь от жильцов, которые оценивают уровень комфорта и безопасности. Далее производится анализ жалоб и пожеланий для того, чтобы повышать эффективность управления и уровень условий проживания в будущем.
            
            Таким образом, функционирование общежития — это сложный многозадачный процесс, требующий внимательности, гибкости и высокой степени координации от административно-технического персонала. Успех (комфортное проживание) зависит от качественного выполнения всех этапов, начиная от заселения и заканчивая анализом жалоб. Только при условии четкой организации работы всех служб и взаимодействия с жильцами можно достичь главной цели — обеспечить безопасный и приемлемый быт для временного населения.
        
%%%%%%%%%%%%%%%%%%%%%%%%%%%%%%%%%%%%%%%%%%%%%%%%%%%%%%%%%%%%%%%
    \newpage
    
    \section{Раздел «Проектирование»}
%%%%%%%%%%%%%%%%%%%%%%%%%%%%%%%%%%%%%%%%%%%%%%%%%%%%%%%%%%%%%%%
    \newpage
    
    \section{Раздел «Программирование»}
%%%%%%%%%%%%%%%%%%%%%%%%%%%%%%%%%%%%%%%%%%%%%%%%%%%%%%%%%%%%%%%
    \newpage
    
    \section*{Заключение}
    \addcontentsline{toc}{section}{Заключение}
%%%%%%%%%%%%%%%%%%%%%%%%%%%%%%%%%%%%%%%%%%%%%%%%%%%%%%%%%%%%%%%
    \newpage
    
    \section*{Список используемых источников}
    \addcontentsline{toc}{section}{Список используемых источников}
%%%%%%%%%%%%%%%%%%%%%%%%%%%%%%%%%%%%%%%%%%%%%%%%%%%%%%%%%%%%%%%
    \newpage
    
    \section*{Приложение А «Исходный код программы генерации схемы базы данных»}
    \addcontentsline{toc}{section}{Приложение А «Исходный код программы генерации схемы базы данных»}
%%%%%%%%%%%%%%%%%%%%%%%%%%%%%%%%%%%%%%%%%%%%%%%%%%%%%%%%%%%%%%%
    \newpage
    
    \section*{Приложение Б «Исходный код программы заполнения базы данных тестовыми данными»}
    \addcontentsline{toc}{section}{Приложение Б «Исходный код программы заполнения базы данных тестовыми данными»}
%%%%%%%%%%%%%%%%%%%%%%%%%%%%%%%%%%%%%%%%%%%%%%%%%%%%%%%%%%%%%%%

\end{document}

%%%%%%%%%%%%%%%%%%%%%%%%%%%%%%%%%%%%%%%%%%%%%%%%%%%%%%%%%%%%%%%
